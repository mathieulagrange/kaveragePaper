\documentclass[10pt]{article}

\title{Cover letter of submission PONE-D-17-23730: "Efficient similarity-based data clustering by optimal object to cluster reallocation"}

\begin{document}

\maketitle

\section{Suitability of the manuscript for publication in PLoS ONE}

The paper presents original work about a clustering method that can be considered to mine a wide range of data. The paper focuses on time series performance analysis in a reproducible evaluation framework. Results and availability of processing and experimental code can thus be of high interest for the research community.

\section{Addition to the scientific literature}

The contribution relates to the research about the use of non-metric similarities that are widely used in time series analysis but also in genomic data mining. Use of the classic k-means algorithms is not feasible, and it is thus common to consider the kernel k-means method that, without special care regarding the semi positiveness of the input matrix is not guaranteed to converge.

The proposed k-averages algorithm is proven to converge for non-metric similarities and demonstrated to reduce the need for memory access and lower complexity with similar clustering performance when compared to the kernel k-means algorithm.

\section{Relation of the paper to previously published work}

This contribution follows works of Brian Kulis \& al in \cite{Kulis2008, Dhillon:2007:WGC:1313055.1313291}. The authors consider the kernel k-means for clustering arbitrary similarity matrices. Regularization methods have been proposed in \cite{Roth:2003:OCP:960254.960291} in order to ensure semi positiveness of the input matrix and thus guaranty clustering convergence but that also affects the performance of the algorithm and the quality of the clustering.

\section{Scientists that will be most interested in your study}

The experiments described in this paper will be of interest for scientist in the machine learning and data mining community.

Thanks to the high performance c implementation provided, any scientist with need for efficient clustering of non metric similarity matrices will be interested in this contribution.


\bibliographystyle{abbrv}
\bibliography{bib}

\end{document}
